\documentclass[11pt,a4paper,final]{article}
\usepackage[utf8]{inputenc}
\usepackage[francais]{babel}
\usepackage[T1]{fontenc}
\usepackage{amsmath}
\usepackage{amsfonts}
\usepackage{amssymb}
\usepackage{graphicx}
\usepackage{lmodern}
\usepackage[left=2cm,right=2cm,top=2cm,bottom=2cm]{geometry}
\author{Antoine}
\begin{document}

\section{Contexte}

La régulation de l'impact des ravageurs sur les cultures, ainsi que la pollinisation des cultures qui en dépendent, sont deux services écosystémiques d'importance pour la production agricole, car ils ont une influence directe sur le rendement \cite{oerke_crop_2006,garibaldi_global_2011}.\\

Les travaux en agro-écologie ont permis de mieux comprendre les déterminants de la relation entre dynamique de la biodiversité et délivrance de ces services, que ce soit en termes d'usages des sols ou de pratiques, à différentes échelles \cite{liere_intersection_2017}. Notamment, la présence de différents types d'éléments semi-naturels, comme les îlots boisés, ou encore les bandes fleuries, semble bien corrélée à la diversité et à l'abondance des pollinisateurs comme des auxiliaires de contrôle biologique \cite{chaplin-kramer_meta-analysis_2011, kennedy_global_2013, shackelford_comparison_2013}. De la même façon le niveau de délivrance des fonctions de pollinisation et de contrôle biologique semble corrélé à ces déterminants \cite{liere_intersection_2017}.\\

La découverte de ces relations a permis la création d'outils de modélisation, et notamment des outils de cartographie. Ces outils utilisent des cartes existantes de végétation et les relations évoquées ci-dessus pour exprimer spatialement  "l'offre écologique" en service de contrôle biologique et de pollinisation pour les cultures, sous la forme de potentiels  \cite{zulian_linking_2013} et de la confronter à d'autres données, notamment d'usage des terres ou économiques, permettant d'exprimer la demande pour ces services. Cette confrontation peut ainsi aider à identifier des zones en tension pour ces services, où la demande pourrait excéder l'offre. La spatialisation de l'offre et de la demande de multiples services autorise aussi à questionner le concept de multifonctionnalité, afin d'établir s'il existe des synergies ou des compromis spatiaux entre ces services \cite{mouchet_spatially_2013}.\\

Néanmoins, il existe à l'heure actuelle une diversité de méthodes permettant d'inférer spatialement l'offre et la demande des services de contrôle biologique et de pollinisation. Ces cartographies ont été réalisées à des échelles diverses, de la région au niveau national voire européen. Elles permettent ou non de confronter offre et demande pour un service, ou bien de confronter les deux services entre eux. \\



\section{Problématique et méthodologie}

En l'absence de revue focalisée sur les méthodes de cartographie et leurs résultats pour ces deux services, il s'agira pour l'étudiant.e :\\

\begin{itemize}

\item d'identifier les différents papiers publiés dans la littérature scientifique traitant de la cartographie de ces deux services ('étudiant.e pourra s'appuyer sur un premier travail de collecte de bibliographie)

\item d'établir et utiliser une grille d'analyse explicitant les méthodes utilisées par ces cartographies, ainsi que leurs principaux résultats, afin de les confronter

\item de discuter l'adéquation entre offre et demande pour ces deux services en fonction des résultats évoqués dans ces études, ainsi que la possibilité d'inadéquation entre ces deux services qui pourrait remettre en cause la multifonctionnalité des paysages agricoles
\end{itemize}

\newpage

\bibliographystyle{unsrt}
\bibliography{SujetBib2}

\end{document}