\documentclass[11pt,a4paper,final]{article}
\usepackage[utf8]{inputenc}
\usepackage[francais]{babel}
\usepackage[T1]{fontenc}
\usepackage{amsmath}
\usepackage{amsfonts}
\usepackage{amssymb}
\usepackage{graphicx}
\usepackage{lmodern}
\usepackage[left=2cm,right=2cm,top=2cm,bottom=2cm]{geometry}
\author{Antoine}
\begin{document}

\section{Contexte}

La régulation de l'impact des ravageurs sur les cultures, ainsi que la pollinisation des cultures qui en dépendent, sont deux services écosystémiques d'importance pour la production agricole, car ils ont une influence directe sur le rendement \cite{oerke_crop_2006,garibaldi_global_2011}.\\

Dans la perspective d'une intensification écologique des pratiques agricoles \cite{bommarco_ecological_2013}, il est souvent proposé de substituer à différents intrants, tels que les pesticides ou à l'utilisation de ruches d'abeilles domestiques, un pilotage de ces services de contrôle et de pollinisation par la maîtrise des processus écologiques à l'origine de ces fonctions. Ces processus peuvent être bottom-up, par exemple en réduisant les ressources nécessaires aux dynamiques de croissance des ravageurs, ou en augmentant celles utiles aux pollinisateurs, ou top-down, en ayant recours aux ennemis naturels qui s'attaquent aux ravageurs \cite{brevault_pest_2019}.\\

Il devient de plus en plus clair que ces processus doivent être analysés et pilotés à une échelle souvent plus large que celle de la seule exploitation, pour privilégier plutôt celle du paysage agricole \cite{landis_designing_2017}. En effet, les organismes impliqués sont souvent très mobiles et peuvent se déplacer sur plusieurs kilomètres, y compris les insectes jouant le rôle de pollinisateurs ou d'ennemis naturels, et les dynamiques de populations ainsi que les réseaux d'interactions qui se forment sont fortement influencés par des mécanismes se réalisant à cette échelle \cite{kremen_pollination_2007}. \\

Envisager une action à l'échelle du paysage agricole implique souvent la mise en place d'une action concertée et collective, car les dimensions d'un "paysage agricole" recouvrent généralement plusieurs exploitations. Une gestion à cette échelle, comparée à une gestion individuelle à l'échelle de l'exploitation, est potentiellement très profitable, spécifiquement pour les services de régulation des impacts des ravageurs et de pollinisation \cite{stallman_ecosystem_2011}. Néanmoins, elle ouvre aussi à des difficultés de conception ainsi que d'implémentation, compte-tenu du fonctionnement et des modes de prises de décision généralement admis dans de nombreuses exploitations agricoles : non-reconnaissance de la dépendance au service \cite{smith_ecosystem_2014}, faible ou mauvaise perception des ressources mobilisables pour le pilotage de ces services \cite{salliou_landscape_2017}, hétérogénéité et incertitude du gain attendu \cite{stallman_determinants_2015}...\\

Malgré ces difficultés, l'idée a fait son chemin et a donné naissance à différents concepts / expressions dans la littérature :
\begin{itemize}
\medskip 
\item "areawide pest management" \cite{brevault_pest_2019},
\item "pest suppressive landscape" \cite{parry_use_2012},
\item "agricultural landscape design" \cite{landis_designing_2017},
\item "agricultural landscape management"
\medskip 
\end{itemize}

ainsi qu'à des efforts concrets de mise en oeuvre d'une conception / implémentation d'une gestion concertée à l'échelle paysage (Landscape design \cite{steingrover_designing_2010} ; Areawide pest management \cite{giles_areawide_2008}) \\

Cependant, de telles initiatives restent rares et semblent se concentrer sur le service de contrôle des ravageurs plutôt que sur le pilotage du service de pollinisation. Cela peut sembler à première vue paradoxal, car d'une part les effets du paysage sur les pollinisateurs sont bien documentés avec des effets plus clairs de la complexité du paysage sur le service de pollinisation \cite{kennedy_global_2013, karp_crop_2018}, en comparaison du service de contrôle biologique, et d'autre part le contrôle biologique est souvent perçu comme un contributeur plus faible de la production agricole que la pollinisation \cite{zhang_european_2018}. \\

Enfin, alors que la notion de paysage agricole multifonctionnel existe dans la littérature depuis longtemps \cite{hatt_spatial_2018}, la gestion conjointe de la pollinisation et du contrôle par une action collective à une telle échelle semble peu mise en œuvre, malgré son potentiel \cite{dainese_high_2017}.

\section{Problématique et méthodologie}

Il s'agira donc pour l'étudiant, et en fonction du temps dont il dispose et des difficultés rencontrées, de :

\subsection{Définir les concepts}

Préciser les définitions et les contours de chacun des 3 concepts ci-dessus et expliciter s'il s'agit réellement de concepts différents, ou simplement d'expressions différentes pour désigner le même objet. Délimiter les communautés de recherche ayant recours à ces différents concepts / expressions.\\

On vérifiera notamment les hypothèses suivantes :

\begin{itemize}
\medskip 
\item ces différentes expressions ne sont pas apparues en même temps dans la littérature
\item elles sont manipulées par des communautés de recherche différente, qu'il est possible de délimiter
\item elles sont utilisées dans des contextes différents
\end{itemize}


\subsection{Creuser le cas du pilotage de la pollinisation}

Vérifier et expliquer l'absence ou la faible utilisation de ces concepts pour le pilotage du service de pollinisation.

\subsection{Identification de cas concrets et synthèse}

Identifier dans la littérature scientifique les cas de conception et d'implémentation d'une gestion collective et concertée d'un paysage agricole, pour le pilotage du contrôle des ravageurs et / ou de la pollinisation.\\

Synthétiser ces études sous forme d'un tableau reprenant les principales caractéristiques de l'étude (échelle, aire géographique, durée des mesures...), ses principaux résultats quand à la gestion des 
services (efficacité), ainsi que les principales recommandations pour traiter les questions d'ordre écologiques / sociales.

\subsection{Propositions pour un paysage multifonctionnel}

Croiser ces résultats d'une part pour la régulation des ravageurs, d'autre part pour la pollinisation, afin d'en tirer d'éventuelles leçons pour la gestion d'un paysage multifonctionnel (contrôle + pollinisation)

\newpage

\bibliographystyle{unsrt}
\bibliography{SujetBib1}

\end{document}